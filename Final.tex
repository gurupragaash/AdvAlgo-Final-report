\documentclass[11pt]{article}

\usepackage[letterpaper]{geometry}
\usepackage{amsmath}
\usepackage{amsthm}
\usepackage{amssymb}
%\usepackage[utf8]{inputenc}
\usepackage{url}
\newcommand{\reals}{{\mathbb R}}

\title{An Improved Approximation Algorithm for Multiway Cut\\Gruia Calinescu Howard Karloff Yuval Rabani\\Interim report for CS 6150}
\author{Gurupragaash Annasamy Mani \and Praveen Thiraviya Rathinam}


\begin{document}
\maketitle

\section{TODO}
\begin{itemize}
    \item What is Multiway cut
    \item Why is multiway cut MAXSNP hard ?
    \item what is MAXSNP/SNP ?
    \item Related work - Dalhaus - the below one . check if we should add some more ?
    \item Ford fulkerson for max flow ????
    \item Chopra and Rao [6] and Cunningham [7] develop a polyhedral approach to Multiway Cut ?
    \item Bertisimas non-linear formulation
    \item The case k=2 is not the only polynomially solvable instance of the multiway cut problem. Lovdsz and Cherkasskij show that when c(e) = 1 and  G is Eulerian, then the multiwaycut problem is polynomially solvable. 
    \item write about simplex, polytope
\end{itemize}
\section{Understanding}
Isolation Heuristic - 
In isolation heuristic, if there are K terminals, in each iteration we attach one terminal to the source and all the other terminals to the sink and then run a max-flow and and find the min-cut. Let the edge set after each iteration be $E_i$. Now the lowest K-1 cuts are taken and the union of them gives the multi-way cut. This algorithm gives an approximation of $2(1-\frac{1}{k})$.
Proof:
\begin{itemize}
    \item Run the isolation heuristic and get the set of edges. Let them be A.
    \item Lets say $E^*$ is the optimal edge set for the multiway cut with K terminals. Then it means if $E^*$ is removed then there will be $K$ disjoint connected graphs($V_1, V_2,\cdots,V_k$), each having one terminal respectively
    \item Lets say $E^* = \sum{i=1}^{k}{E_i^*}$ and each $E_i^*$ represents the edges removed to disconnect $V_i$ from the rest.
    \item Lets say $\delta(V_i)$ gives the set of all outgoing edges from $V_i$.
    \item Now we can say that, $w(E_i) \le w(\delta(V_i))$ because, both of isolates the terminal $i$ from the rest and we know $E_i$ is the mincut. So $w(E_i)$ cannot be greater than $w(\delta(V_i))$
    \item Lets says there is $m$ edges between $V_i$ and $V_{i+1}$. Then both $\delta(V_i)$ and $\delta(V_{i+1})$ will include those $m$ edges.
    \item $2w(E^*) = \sum_{i=1}{k}{\delta(V_i)} $. Its 2 times since each edges is double counted. 
    \item We know $w(A) \le (1 -\frac{1}{k})(\sum_{i=1}(E_i^*))$. This is because, A is the union of first K-1 smallest set. In this expression, we are adding up all the $K$ values. Since only $k-1$ values are taken, we are multiplying it with $1 - \frac{1}{k}$ and since it the union of all these set, it will be equal to or less than the summation of indiviual sets.
        \begin{align*}
            w(A) &\le (1 -\frac{1}{k})(\sum_{i=1}(E_i^*)) \\
            &\le (1 -\frac{1}{k})(\sum_{i=1}{\delta(V_i)}\\
            &\le 2(1 -\frac{1}{k})w(E^*)\\
            &\le 2(1 -\frac{1}{k}) OPT
        \end{align*}

\end{itemize}
How did the authors comeup with the Linear program which they say is an SLP.
\begin{itemize}
    \item As discussed earlier, any valid solution will split the Graphi $G(V,E)$ into $K$ graphs($C_1, C_2,\cdots,C_k$), with each $C_i$ having a terminal $S_i$
    \item Let $\delta(C_i)$ represent the set of edges which goes out of $C_i$
    \item Each vertex $v \in V$ is represented as $x_i^j$ and this value is set to 1 if the $i^{th}$ vertex is part of the componenet $C_i$, else its set to 0. Since a vertex can be a part of only one component, we can say that $\sum_{j=1}^{k}{x^j} = 1, \forall x \in V$, 
    \item We define another notation $z_e^i$ which operates on all the edges in E and is set to 1 if the edge $e$ is in $\delta(C_i)$ else its set to 0
    \item If $e$ is in the $\delta(C_i)$, then it connect $(u,v)$ where $u \in C_i, v \notin C_i$. Thus we can write $z_e^i$ as $z_e^i = x_u^i - x_v^i$, since $x_u^i$ will be 1 and $x_v^i$ will be 0, as per the definition of $x$. If $e$ is not in the $\delta(C_i)$, then both u and v belong to a same component, then both $x_u^i$ and $x_v^i$ will be 1 if both belong to component $C_i$ else both will be 0.
    \item Now for the objective function is $\frac{1}{2}\sum\limits_{e \in E}{c_e * \sum\limits_{i=1}^{k}{z_e^i}}$, where $c_e$ is the cost/weight of the edge. We are multiplying it with $\frac{1}{2}$ because of the double counting which we discussed in the previous explanation.
    \item Thus the LP is 
        \begin{align*}
            \min \quad & \frac{1}{2}\sum\limits_{e \in E}{c_e * \sum\limits_{i=1}^{k}{z_e^i}}\\
            \text{such that}  \quad z_e^i &= |x_u^i - x_v^i|, \quad \forall e \in E\\
            \sum\limits_{i=1}^{k}{x_v^i} &= 1,\quad \forall v \in V\\
            x_{s_i}^i &= 1,\quad \forall s_i \in T\text{(Set of Terminals)}\\
            x_v^j &\ge 0, \quad \forall v \in V\quad (\text{Got by adding LP relaxation to }x_v^j \in (0,1),\quad \forall v \in V)\\
        \end{align*}
\end{itemize}
\end{document}
